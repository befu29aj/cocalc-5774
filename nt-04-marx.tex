% nt-04-marx.tex

\documentclass[xcolor=dvipsnames]{beamer}
\usepackage{teachbeamer}

\title{Nietzsche and Marx}
\subtitle{{\CourseNumber}, {\CourseInst}}

\author{\CourseName}

\date{June 5, 2018}

\begin{document}

\begin{frame}
  \titlepage
\end{frame}

\begin{frame}
  \frametitle{Some Terms}
  \begin{description}
  \item[dialectical materialism] materialism: explanation in terms of
    material things (not ideas) -- the explanatory primacy of economic
    production; dialectic: explanation always depends on history (note
    Engels' ``doubt as to the eternal validity of that which exists'')
  \item[realism] contrasted with dramatizations, idealizations, class
    distortions (Engels insists that realism is sufficient and
    pedagogy unnecessary); Manifesto: ``man is at last compelled to
    face with sober senses his real conditions of life and his
    relations with his kind'' (2)
  \item[bourgeoisie] the class between the aristocracy and the
    proletariat
  \end{description}
\end{frame}

\begin{frame}
  \frametitle{iClicker Question}
Choose from the following options. This item will be graded.
\begin{block}{iClicker Question}
   Which of these phrases occurs in the Manifesto?
\end{block}
\begin{description}
% W. Churchill
\item[A\hspace{.2in}$\blacktriangleright$] Success consists of going from failure to failure without loss of enthusiasm.
% B. Sanders
\item[B\hspace{.2in}$\blacktriangleright$] Let us wage a moral and political war against the billionaires and corporate leaders, whose policies and greed are destroying the middle class.
% K. Marx
\item[C\hspace{.2in}$\blacktriangleright$] Two great hostile camps facing each other: Bourgeoisie and Proletariat.
% A. Lincoln
\item[D\hspace{.2in}$\blacktriangleright$] Give me six hours to chop down a tree and I will spend the first four sharpening the axe.
\end{description}
\end{frame}

\begin{frame}
  \frametitle{Eternal Truths}
  \begin{quote}
    Communism abolishes eternal truths, it abolishes all religion, and
    all morality, instead of constituting them on a new basis. (10)
  \end{quote}
  \begin{quote}
    Law, morality, religion, are to [the proletarian] so many
    bourgeois prejudices, behind which lurk in ambush just as many
    bourgeois interests. (6) (Compare Bernard Mandeville's \emph{The
      Fable of the Bees}.)
  \end{quote}
\end{frame}

\begin{frame}
  \frametitle{School}
  \begin{quote}
And your education! Is not that also social, and determined by social
conditions under which you educate, by the intervention, direct or
indirect, of society, by means of schools? (9)
  \end{quote}
\end{frame}

\begin{frame}
  \frametitle{Family}
  \begin{quote}
    The bourgeois claptrap about the family and education, about the
    hallowed co-relation of parent and child, becomes all the more
    disgusting, the more, by the action of modern industry, all family
    ties among the proletarians are torn asunder, and their children
    transformed into simple articles of commerce and instruments of
    labour. (9)
  \end{quote}
\end{frame}

\begin{frame}
  \frametitle{A Trump Prophecy in the Manifesto}
  \begin{quote}
    They direct their attacks not against the bourgeois conditions of
    production, but against the instruments of production themselves;
    they destroy imported wares that compete with their labour, they
    smash to pieces machinery, they set factories ablaze, they seek to
    restore by force the vanished status of the workman of the Middle
    Ages. (5)
  \end{quote}
\end{frame}

\begin{frame}
  \frametitle{Themes from Manifesto of the Communist Party}
  \begin{itemize}
  \item<1-> freedom vs. free trade; ``in bourgeois society capital is
    independent and has individuality, while the living person is
    dependent and has no individuality'' (8); ``the free development
    of each is the condition for the free development of all'' (11)
  \item<2-> the bourgeoisie is self-undermining (4) because it depends
    on unlimited growth
  \end{itemize}
\end{frame}

\end{document}
