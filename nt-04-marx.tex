% nt-04-marx.tex

\documentclass[xcolor=dvipsnames]{beamer}
\usepackage{teachbeamer}

\title{Nietzsche and Marx}
\subtitle{{\CourseNumber}, {\CourseInst}}

\author{\CourseName}

\date{June 5, 2018}

\begin{document}

\begin{frame}
  \titlepage
\end{frame}

% \begin{frame}
%   \frametitle{iClicker Question}
% Choose from the following options. This item will be graded.
% \begin{block}{iClicker Question}
% [3075] Which of these phrases occurs in the Manifesto?
% \end{block}
% \begin{description}
% % W. Churchill
% \item[A\hspace{.2in}$\blacktriangleright$] Success consists of going from failure to failure without loss of enthusiasm.
% % B. Sanders
% \item[B\hspace{.2in}$\blacktriangleright$] Let us wage a moral and political war against the billionaires and corporate leaders, whose policies and greed are destroying the middle class.
% % K. Marx
% \item[C\hspace{.2in}$\blacktriangleright$] Two great hostile camps facing each other: Bourgeoisie and Proletariat.
% % A. Lincoln
% \item[D\hspace{.2in}$\blacktriangleright$] Give me six hours to chop down a tree and I will spend the first four sharpening the axe.
% \end{description}
% \end{frame}

% \begin{frame}
%   \frametitle{iClicker Question}
% Choose from the following options. This item will be graded.
% \begin{block}{iClicker Question}
% [7174] In which two cities did Karl Marx live because he was expelled from Germany on political grounds?
% \end{block}
% \begin{description}
% \item[A\hspace{.2in}$\blacktriangleright$] Prague and Moscow
% \item[B\hspace{.2in}$\blacktriangleright$] Madrid and Budapest
% \item[C\hspace{.2in}$\blacktriangleright$] Vienna and Rome
% \item[D\hspace{.2in}$\blacktriangleright$] London and Brussels
% \end{description}
% \end{frame}

% \begin{frame}
%   \frametitle{iClicker Question}
% Choose from the following options. This item will be graded.
% \begin{block}{iClicker Question}
% [3268] Who, according to R. Jay Wallace, are ``the contemporary successors to Nietzsche's ascetic priests''?
% \end{block}
% \begin{description}
% \item[A\hspace{.2in}$\blacktriangleright$] the populist politicians, preachers, and imams of revenge
% \item[B\hspace{.2in}$\blacktriangleright$] the new atheists
% \item[C\hspace{.2in}$\blacktriangleright$] the scientific elite: doctors and academics
% \item[D\hspace{.2in}$\blacktriangleright$] the billionaire class
% \end{description}
% \end{frame}

% \begin{frame}
%   \frametitle{iClicker Question}
% Choose from the following options. This item will be graded.
% \begin{block}{iClicker Question}
% [9839] Which two explanations of the slave revolt are in tension in R. Jay Wallace's paper?
% \end{block}
% \begin{description}
% \item[A\hspace{.2in}$\blacktriangleright$] the strategic and the expressive interpretation
% \item[B\hspace{.2in}$\blacktriangleright$] the minimalist and the maximalist interpretation
% \item[C\hspace{.2in}$\blacktriangleright$] the exogenous and the endogenous interpretation
% \item[D\hspace{.2in}$\blacktriangleright$] the hermeneutic and the scientific interpretation
% \end{description}
% \end{frame}

\begin{frame}
  \frametitle{Some Terms}
  \begin{description}
  \item[dialectical materialism] materialism: explanation in terms of
    material things (not ideas) -- the explanatory primacy of economic
    production; dialectic: explanation always depends on history (note
    Engels' ``doubt as to the eternal validity of that which exists'')
  \item[realism] contrasted with dramatizations, idealizations, class
    distortions (Engels insists that realism is sufficient and
    pedagogy unnecessary); Manifesto: ``man is at last compelled to
    face with sober senses his real conditions of life and his
    relations with his kind'' (2)
  \item[bourgeoisie] the class between the aristocracy and the
    proletariat
  \end{description}
\end{frame}

\begin{frame}
  \frametitle{Commodity Fetishism}
  In Marxism, \alert{reification} is a process by which objects (such
  as commodities) turn into subjects and subjects (such as people)
  turn into objects (means of production). What once was an object
  becomes by reification active and determining, while the subject
  becomes passive and determined. My namesake Gy{\"o}rgy Luk{\'a}cs
  says in \emph{History and Class Consciousness}
  \begin{quote}
    Just as the capitalist system continuously produces and reproduces
    itself economically on higher levels, the structure of reification
    progressively sinks more deeply, more fatefully, and more
    definitively into the consciousness of man.
  \end{quote}
\end{frame}

\begin{frame}
  \frametitle{Reification of Consciousness}
  Here is an important example of reification: the reification of
  consciousness.
  \begin{block}{Karl Marx}
    From this moment onwards consciousness can really flatter itself
    that it is something other than consciousness of existing
    practice, that it really represents something without representing
    something real; from now on consciousness is in a position to
    emancipate itself from the world and to proceed to the formation
    of ``pure'' theory, theology, philosophy, ethics, etc.
  \end{block}
  Nietzsche often thinks of this process as making the false inference
  from a deed to a doer; from a verb to a subject; from thought to
  thinker. Both Marx and Nietzsche are suspicious of language and its
  grammar. Marx says, for example, that ``for the bourgeois it is so
  much the easier to prove on the basis of his language {\ldots} since
  this language itself is a product of the bourgeoisie'' (Love, 109).
\end{frame}

\begin{frame}
  \frametitle{Preface to \emph{A Contribution to the Critique of
        Political Economy}}
  \begin{quote}
    In the social production of their existence, men inevitably enter
    into definite relations, which are independent of their will,
    namely relations of production appropriate to a given stage in the
    development of their material forces of production. The totality
    of these relations of production constitutes the economic
    structure of society, the real foundation, on which arises a legal
    and political superstructure and to which correspond definite
    forms of social consciousness. The mode of production of material
    life conditions the general process of social, political and
    intellectual life {\ldots}
  \end{quote}
\end{frame}

\begin{frame}
  \frametitle{Preface to \emph{A Contribution to the Critique of
        Political Economy}}
  \begin{quote}
    {\ldots} It is not the consciousness of men that determines their
    existence, but their social existence that determines their
    consciousness. At a certain stage of development, the material
    productive forces of society come into conflict with the existing
    relations of production or – this merely expresses the same thing
    in legal terms – with the property relations within the framework
    of which they have operated hitherto. From forms of development of
    the productive forces these relations turn into their fetters.
    Then begins an era of social revolution. The changes in the
    economic foundation lead sooner or later to the transformation of
    the whole immense superstructure.
  \end{quote}
\end{frame}

\begin{frame}
  \frametitle{Commodity Fetishism}
  Reification may be the consequence of a simple fallacy:
  \alert{hypostatization}, where an abstraction is treated as if it
  were concrete. For example, you may for a moment mistake the map for
  the territory it represents. Reification leads to
  \alert{alienation}, for example the alienation of the worker from
  production.
\end{frame}

\begin{frame}
  \frametitle{What the Bourgeoisie Has Accomplished}
The bourgeoisie has
  \begin{itemize}
  \item resolved personal worth into exchange value
  \item reduced the family relation to a mere money relation
  \item made itself dependent on permanent revolution
  \item rescued a considerable part of the population from the idiocy
    of rural life
  \item agglomerated production and concentrated property in a few
    hands
  \item has undermined itself by creating the proletariat (``what the
    bourgeoisie produces is its own grave-diggers'')
  \end{itemize}
\end{frame}

\begin{frame}
  \frametitle{Eternal Truths}
  \begin{quote}
    Communism abolishes eternal truths, it abolishes all religion, and
    all morality, instead of constituting them on a new basis. (10)
  \end{quote}
  \begin{quote}
    Law, morality, religion, are to [the proletarian] so many
    bourgeois prejudices, behind which lurk in ambush just as many
    bourgeois interests. (6)
  \end{quote}
\end{frame}

\begin{frame}
  \frametitle{The Fable of the Bees}
  Here is an interesting precursor to Marxian and Nietzschean ideas
  about morality: Bernard Mandeville's \emph{The Fable of the Bees}.
  \begin{itemize}
  \item virtue and morality have as their primary function to
    lubricate the economy (Mandeville therefore insists that they are
    necessary)
  \item vice and immorality are also necessary for the economy, for
    example in inducing us to more spending and consumption
  \end{itemize}
  Rather than being opposed to each other, virtue and vice are
  carefully balanced and conditioned on each other by the requirements
  of a functioning economy.
\end{frame}

\begin{frame}
  \frametitle{Communist Manifesto}
  The aim of communists is
  \begin{itemize}
  \item the formation of the proletariat into a class
  \item the overthrow of the bourgeois supremacy
  \item the conquest of political power by the proletariat
  \item the use of accumulated labour to widen, to enrich, and to
    promote the existence of the labourer
  \end{itemize}
  The abolition of existing property relations is a consequence of the
  historical process, not unusual in other epochs, and not an explicit
  goal of communists. Communists seek to abolish private property,
  i.e.\ bourgeois property (just as the bourgeois French revolution
  sought to expropriate feudal property). 
\end{frame}

\begin{frame}
  \frametitle{Communist Manifesto}
  Just as for Nietzsche, morality and its metaphysical manifestations
  (freedom, law, responsibility) is based on a historical
  psychological development; it based on a historical economic
  development in the Communist Manifesto:
  \begin{quote}
    the standard of your bourgeois notions of freedom, culture, law,
    etc. {\ldots} are but the outgrowth of the conditions of your
    bourgeois production and bourgeois property
  \end{quote}
  Aristocrats in their reactionary and charitable support for the
  cause of the proletariat (having the bourgeoisie as a common enemy)
  stoop to ``barter truth, love, and honour for traffic in wool,
  beetroot-sugar, and potato spirits.''
\end{frame}

\begin{frame}
  \frametitle{Communist Manifesto}
  \begin{block}{consciousness and material existence}
    man's ideas, views, and conceptions, in one word man's
    consciousness, changes with every change in the conditions of his
    material existence, in his social relations and in his social life
  \end{block}
\end{frame}

\begin{frame}
  \frametitle{School}
  \begin{quote}
And your education! Is not that also social, and determined by social
conditions under which you educate, by the intervention, direct or
indirect, of society, by means of schools? (9)
  \end{quote}
\end{frame}

\begin{frame}
  \frametitle{Family}
  \begin{quote}
    The bourgeois claptrap about the family and education, about the
    hallowed co-relation of parent and child, becomes all the more
    disgusting, the more, by the action of modern industry, all family
    ties among the proletarians are torn asunder, and their children
    transformed into simple articles of commerce and instruments of
    labour. (9)
  \end{quote}
\end{frame}

\begin{frame}
  \frametitle{Abolition}
  Marx and Engels address the following fears of the bourgeoisie with
  respect to communism.
  \begin{itemize}
  \item abolition of property $\longrightarrow$ abolition of private
    (bourgeois) property
  \item abolition of monogamy $\longrightarrow$ abolition of public
    and private prostitution
  \item abolition of home education $\longrightarrow$ abolition of
    bourgeois social intervention in education
  \item abolition of countries $\longrightarrow$ bourgeoisie and
    proletariat both abolish nationalism, the proletariat will do so
    achieving peace
  \item abolition of eternal truths $\longrightarrow$ class
    antagonisms have indeed been a permanent feature of social
    relations, the communist revolution promises the most radical
    rupture
  \end{itemize}
\end{frame}

\begin{frame}
  \frametitle{Communist Manifesto}
  Here is the Communist Manifesto's political action plan:
  \begin{enumerate}
  \item raise the proletariat to the position of ruling as to win the
    battle of democracy (contra Lenin)
  \item use the proletariat's political supremacy to wrest, by degrees
    (!), all capital from the bourgeoisie (this cannot be effected
    except by means of despotic inroads on the rights of property)
  \item centralize all instruments of production in the hands of the
    State, i.e.\ of the proletariat organized as the ruling class
  \item increase the total of productive forces as rapidly as possible
  \end{enumerate}
\end{frame}

\begin{frame}
  \frametitle{The Communist Decalogue}
  \begin{enumerate}
  \item Abolition of landed property
  \item A heavy progressive income tax
  \item Abolition of rights of inheritance
  \item Confiscation of rebel/emigrant property
  \item Centralization of credit in the hands of the State
  \item Centralization of communication/transport in the hands of the
    State
  \item Extension of State production
  \item Equal liability of all to labour (establishment of industrial
    armies)
  \item Combination of agriculture with manufacturing, more equable
    distribution of population
  \item Free education (abolish children's factory labour, but combine
    education with industrial production)
  \end{enumerate}
\end{frame}

\begin{frame}
  \frametitle{Communist Manifesto}
  In the last two paragraphs of Section II, Marx and Engels predict
  that class distinctions will dissolve, ``public power will lose its
  political character,'' and ``the free development of each is the
  condition for the free development of all.'' This prediction
  strongly clashes with the descriptive projects of Nietzsche and
  Foucault (and certainly with Nietzsche's normative ideas about power
  differentials).
\end{frame}

\begin{frame}
  \frametitle{Communist Manifesto}
  Marx and Engels reject other communist/socialist literature mainly
  on a Nietzschean basis: those other communists/socialists seek to
  improve the lot of every class, not just the lot of the proletariat.
  Sometimes they do so (perhaps subconsciously) in order to further
  the ends of their own class (``the old feudal coat of arms on their
  hindquarters'').
  \begin{description}
  \item[reactionary socialism] the feudal aristocracy using the
    proletariat in its resentment against the bourgeoisie
  \item[petty bourgeois socialism] Bernie Sanders type socialism
  \item[German socialism] ``not the interests of the proletariat, but
    the interests of Human Nature, of Man in general, who belongs to
    no class, has no reality, who exists only in the misty realm of
    philosophical fantasy'' (compare Taylor's critique of
    existentialism)
  \item[bourgeois socialism] they wish for a bourgeoisie without
    proletariat (and got it in the developed world, didn't they?)
  \end{description}
\end{frame}

\begin{frame}
  \frametitle{A Trump Prophecy in the Manifesto}
  \begin{quote}
    They direct their attacks not against the bourgeois conditions of
    production, but against the instruments of production themselves;
    they destroy imported wares that compete with their labour, they
    smash to pieces machinery, they set factories ablaze, they seek to
    restore by force the vanished status of the workman of the Middle
    Ages. (5)
  \end{quote}
\end{frame}

\begin{frame}
  \frametitle{Themes from Manifesto of the Communist Party}
  \begin{itemize}
  \item<1-> freedom vs. free trade; ``in bourgeois society capital is
    independent and has individuality, while the living person is
    dependent and has no individuality'' (8); ``the free development
    of each is the condition for the free development of all'' (11)
  \item<2-> the bourgeoisie is self-undermining (4) because it depends
    on unlimited growth
  \end{itemize}
\end{frame}

% \begin{frame}
%   \frametitle{iClicker Question}
% Choose from the following options. This item will be graded.
% \begin{block}{iClicker Question}
% [8170] Who, according to Nietzsche, has made the only attempt so far to write a history of the emergence of morality?
% \end{block}
% \begin{description}
% \item[A\hspace{.2in}$\blacktriangleright$] Russian anarchists
% \item[B\hspace{.2in}$\blacktriangleright$] Italian revolutionaries
% \item[C\hspace{.2in}$\blacktriangleright$] English psychologists
% \item[D\hspace{.2in}$\blacktriangleright$] Austrian aristocrats
% \end{description}
% \end{frame}

% \begin{frame}
%   \frametitle{iClicker Question}
% Choose from the following options. This item will be graded.
% \begin{block}{iClicker Question}
% [8183] What do the bad (i.e.\ the powerless) feel towards the good (i.e.\ the powerful), which by the priestly creation of values then turns those who are good into those who are evil?
% \end{block}
% \begin{description}
% \item[A\hspace{.2in}$\blacktriangleright$] chutzpah
% \item[B\hspace{.2in}$\blacktriangleright$] ressentiment
% \item[C\hspace{.2in}$\blacktriangleright$] panache
% \item[D\hspace{.2in}$\blacktriangleright$] transference
% \end{description}
% \end{frame}\begin{frame}
%   \frametitle{iClicker Question}
% Choose from the following options. This item will be graded.
% \begin{block}{iClicker Question}
%   [4790] Both Nietzsche and Marx, as philosophers, lived in the shadow of the idealist philosopher GWF Hegel. They were significantly influenced by him and defined themselves in contrast to him. What, according to Love, is a distinguishing feature of Hegelian philosophy to which both Marx and Nietzsche object?
% \end{block}
% \begin{description}
% \item[A\hspace{.2in}$\blacktriangleright$] its view of animal-human continuity
% \item[B\hspace{.2in}$\blacktriangleright$] its moral claims
% \item[C\hspace{.2in}$\blacktriangleright$] its scientific results
% \item[D\hspace{.2in}$\blacktriangleright$] its teleological explanations
% \end{description}
% \end{frame}

% \begin{frame}
%   \frametitle{iClicker Question}
% Choose from the following options. This item will be graded.
% \begin{block}{iClicker Question}
% [9296] Which of these names occur in the Love reading?
% \end{block}
% \begin{description}
% \item[A\hspace{.2in}$\blacktriangleright$] Althusser, Kaufmann, Foucault, Deleuze
% \item[B\hspace{.2in}$\blacktriangleright$] Chomsky, Russell, Wittgenstein, Popper
% \item[C\hspace{.2in}$\blacktriangleright$] Arendt, Levi-Strauss, Bergson, Derrida
% \item[D\hspace{.2in}$\blacktriangleright$] Anscombe, Taylor, Husserl, Einstein
% \end{description}
% \end{frame}

\begin{frame}
  \frametitle{Dialectical History}
  According to Nancy Love, Marx and Nietzsche have in common that they
  dissolve traditional dichotomies:
  \begin{itemize}
  \item between nature and history
  \item between objective reality and subjective preference (the
    impossibility of the detached observer)
  \item between morality and power; or justice and interest
  \end{itemize}
\end{frame}

\begin{frame}
  \frametitle{Dialectical History}
  Marx and Nietzsche's historical methods are \alert{historical
    materialism} and \alert{genealogy}, respectively. Here is what
  they are not, according to Love:
  \begin{enumerate}
  \item transcendent or immanent teleologies
  \item economic or physiological determinisms
  \item incoherent histories
  \end{enumerate}
\end{frame}

\begin{frame}
  \frametitle{Dialectical History}
  Both Nietzsche and Marx, as philosophers, lived in the shadow of the
  idealist philosopher GWF Hegel. They were significantly influenced
  by him and defined themselves in contrast to him. Hegel viewed
  history as the autogenesis of spirit. Nietzsche and Marx reject
  teleology (making the consequent the antecedent) in favour of causal
  explanation.
  \begin{block}{Nancy Love: Dialectical History}
    Nonetheless, Nietzsche's overman and Marx's communist society are
    often regarded as immanent historical goals. (73)
  \end{block}
  Nietzsche and Marx would counter that making speculative predictions
  does not turn the consequent into an antecedent. It is always an
  interesting question to what extent a philosopher's descriptive
  project is related to a normative theory.
\end{frame}

\begin{frame}
  \frametitle{Dialectical History}
  What is especially problematic about teleology for Nietzsche and
  Marx is that it presupposes that consciousness forms history, when
  precisely the opposite is true. Consciousness is derivative, not
  determinant.

  \bigskip

  Nietzsche and Marx resist determinism in so far as they would always
  explain history as a confluence of a multitude of factors. Foucault
  and Althusser, respectively, bring their opposition to determinism
  into greater relief
\end{frame}

\begin{frame}
  \frametitle{Dialectical History}
  Here is an area of agreement between Marx and Nietzsche:
  \begin{block}{Nancy Love: Dialectical History}
    Men developed social customs because they have an instinct to
    dominate; those customs by creating consciousness then furthered
    the expansion of the will to power. Only ideologists reify conscious
    purposes, separating consciousness as a ``cause'' of history from
    the forces which functionally determine it. (88)
  \end{block}
  Both Christian morality and bourgeois capitalism are
  self-undermining, dynamic historical developments; they are not
  to be hypostatized. Both Marx and Nietzsche do not take the current
  societal configuration to be ``ordained,'' but ask if it is still
  functional for man's expanding powers.
\end{frame}

\begin{frame}
  \frametitle{Dialectical History}
  Here is what's wrong with current economics and social science (as
  apropos today as it was in the 19th century):
  \begin{quote}
    Political economists and scientific historians {\ldots} cannot
    explain or evaluate social wholes. They can only describe given
    realities as eternal; they are the most virulent positivists. (90)
  \end{quote}
  \alert{Positivism:} Society and nature operate according to laws,
  which can be known only on the basis of empirical evidence (not
  introspection or intuitive knowledge) interpreted by logic.
\end{frame}

\begin{frame}
  \frametitle{Dialectical History}
  \begin{block}{Karl Marx: Poverty of Philosophy}
    Economists express the relations of bourgeois production, the
    division of labour, credit, money, etc. as fixed immutable,
    eternal categories {\ldots} economists explain how production
    takes place in the above mentioned relations, but what they do not
    explain is how these relations themselves are produced, that is
    the historical movement that gave them birth {\ldots} these
    categories are as little eternal as the relations they express.
    They are historical and transitory products.
  \end{block}
\end{frame}

\begin{frame}
  \frametitle{Dialectical History}
  Here is what Nietzsche says about extreme descriptivism.
  \begin{block}{Genealogy of Morals, III, 26}
    Its noblest claim nowadays is that it is a mirror; it rejects all
    teleology; it no longer wishes to ``prove'' anything; it disdains
    to play the judge and considers this a sign of good taste---it
    affirms as little as it denies; it ascertains, it ``describes''
    {\ldots} All this is to a high degree ascetic; but at the same
    time it is to an even higher degree nihilistic {\ldots}
  \end{block}
\end{frame}

\begin{frame}
  \frametitle{Dialectical History}
  Engel's Three Basic Laws of Dialectics.
  \begin{enumerate}
  \item transformation of quantity into quality and vice versa
  \item interpenetration of opposites
  \item negation of negation
  \end{enumerate}
\end{frame}

\begin{frame}
  \frametitle{Strategic Interpretation}
  In ``Ressentiment, Value, and Self-Vindication,'' R. Jay Wallace proposes an alternative account to the
  \alert{strategic interpretation}. Here are the problems with the
  strategic interpretation.
  \begin{itemize}
  \item The strategic interpretation presupposes that ressentiment
    leads to a table of new values which will injure the interests of
    the oppressors. Yet if the new values were created
    \alert{instrumentally} they cannot serve \alert{intrinsically} as
    a framework for preference, deliberation, and criticism (114).
  \item Even if the strategic features remain below the level of
    consciousness, it is not clear how instrumental rationality could
    arrive at a successful formula that leads from the creation of the
    new values to the injury for the oppressors.
  \end{itemize}
\end{frame}

\begin{frame}
  \frametitle{Expressive Interpretation}
  \begin{block}{R. Jay Wallace: Ressentiment, Value, and
      Self-Vindication}
    The ur-context of ressentiment is one in which some people have
    things that you very much desire, but that you lack and feel
    yourself unable ever to obtain. (116)
  \end{block}
  \begin{itemize}
  \item status
  \item material possessions
  \item political power
  \end{itemize}
  RJW does not mention sex and cultural visibility. 
\end{frame}

\begin{frame}
  \frametitle{Expressive Interpretation}
  What element is added to envy to result in ressentiment? RJW
  considers systematic barriers to the achievement of the desired
  goods (117). Janaway's perspective may be helpful here: envy plus
  the desire to injure may result in ressentiment, whether or not
  there are systematic barriers.

  \bigskip

  On page 128, RJW appears to be in greater agreement with Janaway in
  so far as ressentiment (in contrast to resentment and envy) is
  primitive and lacks more fundamental value judgments (and can
  therefore lead to new values without undermining itself).
\end{frame}

\begin{frame}
  \frametitle{Expressive Interpretation}
  \begin{block}{R. Jay Wallace: Ressentiment, Value, and
      Self-Vindication}
    The fundamental emotional dynamic of the slave revolt is not the
    selection of means to an end that is set by one's desires. It is
    the expression of one's negative emotional orientation toward the
    powerful, in the embrace of an evaluative framework that makes
    sense of that basic orientation. (118)
  \end{block}
  Question: do the powerless really have negative emotions towards
  their oppressors? Do they not rather admire them? Emulate them (see
  127f for RJW's response)? Is it not rather the powerless who against
  their own interests subscribe to the values of the powerful? Do they
  not rather direct their frustration at their plight against
  themselves (Janaway)?
\end{frame}

\begin{frame}
  \frametitle{Vindication}
  In Janaway, there is cognitive dissonance in the agent who inflicts
  punishment on herself: why is she hurting herself? This dissonance
  is resolved by a migrated concept from debtor-creditor
  relationships (rather than the more accurate explanation of a
  frustrated will to power).

  \bigskip

  In RJW, there is another cognitive dissonance. The powerless person
  asks herself: why do I hate those that are ``good'' (the powerful)?
  The dissonance is resolved by inverting the evaluative spectrum: the
  good (aristocrats) are evil, the bad (powerless) are good. This
  inversion vindicates the pent-up hatred that the powerless focus on
  the powerful.
\end{frame}

\begin{frame}
  \frametitle{Cui Bono?}
  In RJW's Nietzsche interpretation, the aristocrats eventually
  assimilate the values of the slave revolt based on the sheer number
  of people who have internalized those values. I have an alternative
  proposal: see next slide.

  \bigskip

  The more relevant question is why does the aristocracy care so much
  about the powerless, not why the powerless hate the aristocracy so
  much. The answer, I suspect, is found in the interests of the
  powerful. Thus, the economic advantage of the wealthy is aligned
  with the implementation of Christian charity. Furthermore, there is
  no Wallace Vindication needed for slaves with respect to the new
  values of the slave revolt---because those also were imposed on them
  by the aristocracy (this is more of a Marxian than a Nietzschean
  viewpoint). 
\end{frame}

\begin{frame}
  \frametitle{Cui Bono?}
  \begin{itemize}
  \item The powerful accepted the values of the slave revolt because
    these values ultimately promote the goals \emph{of the powerful}.
    Political and cultural liberalism has succeeded more effectively
    at creating a superior class than uncouth conservativism (treating
    the poor as objects of disdain rather than charity), which has
    never been able to harness the productive powers of the oppressed.
\item It is the invention of the ``middle class'' that has given
  almost unlimited potential of wealth and power accumulation to the
  currently ruling aristocracy.
\item The question is whether the aristocracy was able to pull off
  this stunt of making themselves loved and served by the masses as a
  result of strategy or the convolutions of historical contingency.
\end{itemize}
\end{frame}

\end{document}
