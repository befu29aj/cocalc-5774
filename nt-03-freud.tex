% nt-03-freud.tex

\documentclass[xcolor=dvipsnames]{beamer}
\usepackage{teachbeamer}

\title{Nietzsche and Freud}
\subtitle{{\CourseNumber}, {\CourseInst}}

\author{\CourseName}

\date{May 29, 2018}

\begin{document}

\begin{frame}
  \titlepage
\end{frame}

\begin{frame}
  \frametitle{iClicker Question}
Choose from the following options. This item will be graded.
\begin{block}{iClicker Question}
[2482] Which one of these infantile activities does Freud address at length?
\end{block}
\begin{description}
\item[A\hspace{.2in}$\blacktriangleright$] Bedwetting
\item[B\hspace{.2in}$\blacktriangleright$] Sleepwalking
\item[C\hspace{.2in}$\blacktriangleright$] Thumbsucking
\item[D\hspace{.2in}$\blacktriangleright$] Pottytraining
\end{description}
\end{frame}

\begin{frame}
  \frametitle{iClicker Question}
Choose from the following options. This item will be graded.
\begin{block}{iClicker Question}
[1440] What are Freud's ``psychical dams''?
\end{block}
\begin{description}
\item[A\hspace{.2in}$\blacktriangleright$] ego, id, super-ego
\item[B\hspace{.2in}$\blacktriangleright$] shame, disgust, and morality
\item[C\hspace{.2in}$\blacktriangleright$] dreams, jokes, Freudian slips
\item[D\hspace{.2in}$\blacktriangleright$] sublimatiion, elimination, projection
\end{description}
\end{frame}

\begin{frame}
  \frametitle{iClicker Question}
Choose from the following options. This item will be graded.
\begin{block}{iClicker Question}
[3484] In Mattia Riccardi's paper, Nietzsche is interpreted to claim
that a Newton of psychology would find causal, not teleological,
explanations for behaviour and actions. What, according to Nietzsche,
are the components of these causal explanations?
\end{block}
\begin{description}
\item[A\hspace{.2in}$\blacktriangleright$] drives
\item[B\hspace{.2in}$\blacktriangleright$] volitions
\item[C\hspace{.2in}$\blacktriangleright$] goals
\item[D\hspace{.2in}$\blacktriangleright$] commands
\end{description}
\end{frame}

\begin{frame}
  \frametitle{iClicker Question}
Choose from the following options. This item will be graded.
\begin{block}{iClicker Question}
[2627] In philosophy, ``privacy'' is the term for access to your own
mental states; ``mindreading'' is the term for access to other
people's mental states. Which of the following is closest to
Nietzsche's view, according to Mattia Riccardi's paper?
\end{block}
\begin{description}
\item[A\hspace{.2in}$\blacktriangleright$] privacy is transparent, mindreading is obscured
\item[B\hspace{.2in}$\blacktriangleright$] privacy gives us privileged access to our own mental states
\item[C\hspace{.2in}$\blacktriangleright$] mindreading is possible only on the basis of privacy
\item[D\hspace{.2in}$\blacktriangleright$] privacy is as obscured as mindreading
\end{description}
\end{frame}

\begin{frame}
  \frametitle{Psycho-Analysis} 
  Many of the things that Freud said were pre-empted by Nietzsche, for
  example the idea of the subconscious. Both Freud and Nietzsche were
  thinkers who were trying to understand what an honest intellectual
  response to the historical phenomenon of modernity could look like.
  Modernity is where
\begin{quote}
  {\ldots} the people have lost their ancient beliefs; the parson sits
  at home and unravels his vestments, one after another {\ldots}
  (Franz Kafka, A Country Doctor, 141)
\end{quote}
\end{frame}

\begin{frame}
  \frametitle{Markers of Modernity} 
  \begin{description}
  \item[epistemological crisis] ``I was in great perplexity'' (136)
  \item[the efficiency of aimlessness] ``you never know what you are
    going to find in your own house'' (137)
  \item[the twinning of pathology and sex] the boy's wound (141)
    Kafka's bisexuality? See Foucault's \emph{History of Sexuality}
  \item[lack of agency] first, the doctor cannot help because the boy
    is healthy; then he cannot help because the boy is past helping (141)
  \item[collapse of eschatology] ``it cannot be made good, not ever''
    (143)
  \end{description}
\end{frame}

\begin{frame}
  \frametitle{Ontogenesis vs. Phylogenesis} Another parallel between
  Nietzsche and Freud is that both were trying to do to the subject
  matter of explanation what Darwin did with respect to life: explain
  by revealing its history. The important difference for Freud is that
  he used \alert{ontogenesis} instead of \alert{phylogenesis} for his
  explanation (25). Here are some key concepts in psycho-analysis.
\end{frame}

\begin{frame}
  \frametitle{Key Concepts I} 
  \begin{description}
  \item[drives] irrational drives determine human behaviour; rational
    explanations are epiphenomenal (confabulation)
  \item[neurosis] conflict between the unconscious and the conscious
    creates repression
  \item[subconscious] the wall between the unconscious and the
    conscious is porous, but information which passes through is
    encrypted in symbols (dreams, myths, jokes, Freudian slips)
  \item[psycho-analysis] therapy is bringing-to-consciousness and
    transference (32)
  \end{description}
\end{frame}

\begin{frame}
  \frametitle{Key Concepts II} 
  \begin{description}
  \item[Oedipus Complex] early childhood amnesia obscures the Oedipus
    complex
  \item[Id Ego Superego] the rational identity is confronted with an
    animalic identity and with a repressive identity (double object
    selection, separated by latency, 37) (for ``id'' see the groom in
    ``A Country Doctor'' (137), for ``superego'' see the priest in ``The
    Trial'')
  \item[sex] the explanatory power of sex (16), although sexual desire
    ultimately must be sublimated (19, 26) (and, sometimes, a cigar is
    just a cigar, 40)
  \end{description}
\end{frame}

\begin{frame}
  \frametitle{Highlights of Infantile Sexuality} 
  \begin{itemize}
  \item sexual innocence and exaggerated sexual desire (16)
  \item notice how Freud struggles to define abnormality (19, see his
    dam analogy on page 26)
  \end{itemize}
\end{frame}

\end{document}
