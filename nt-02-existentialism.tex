% nt-02-existentialism.tex

\documentclass[xcolor=dvipsnames]{beamer}
\usepackage{teachbeamer}

\title{Narrative Identity}
\subtitle{{\CourseNumber}, {\CourseInst}}

\author{\CourseName}

\date{May 22, 2018}

\begin{document}

\begin{frame}
  \titlepage
\end{frame}

\begin{frame}
  \frametitle{iClicker Question}
Choose from the following options. This item will be graded.
\begin{block}{iClicker Question}
[4724] What is the cornerstone of existentialism?
\end{block}
\begin{description}
\item[A\hspace{.2in}$\blacktriangleright$] man's existence is more firmly established than God's existence
\item[B\hspace{.2in}$\blacktriangleright$] existence precedes essence
\item[C\hspace{.2in}$\blacktriangleright$] man is not free but bound to his existence
\item[D\hspace{.2in}$\blacktriangleright$] everything is morally permissible
\end{description}
\end{frame}

\begin{frame}
  \frametitle{iClicker Question}
Choose from the following options. This item will be graded.
\begin{block}{iClicker Question}
[1387] Which of these are not part of a Sartre illustration in his paper?
\end{block}
\begin{description}
\item[A\hspace{.2in}$\blacktriangleright$] joining the resistance or staying with one's mother
\item[B\hspace{.2in}$\blacktriangleright$] the picture of Dorian Gray
\item[C\hspace{.2in}$\blacktriangleright$] a man who becomes a Jesuit after many setbacks in life
\item[D\hspace{.2in}$\blacktriangleright$] joining a Christian or a Communist trade union
\end{description}
\end{frame}

\begin{frame}
  \frametitle{iClicker Question}
Choose from the following options. This item will be graded.
\begin{block}{iClicker Question}
[4435] Which one of these is a core moral dilemma in Charles Taylor's essay
``What Is Human Agency?''?
\end{block}
\begin{description}
\item[A\hspace{.2in}$\blacktriangleright$] legalizing abortion vs valuing human life
\item[B\hspace{.2in}$\blacktriangleright$] eating organic local food vs giving to charity
\item[C\hspace{.2in}$\blacktriangleright$] persisting in an academic job vs moving to Nepal
\item[D\hspace{.2in}$\blacktriangleright$] keeping one healthy person alive vs having five sick patients die
\end{description}
\end{frame}

\begin{frame}
  \frametitle{iClicker Question}
Choose from the following options. This item will be graded.
\begin{block}{iClicker Question}
[3878] Which of these contrast pairs is not a distinction found in Charles
Taylor's paper?
\end{block}
\begin{description}
\item[A\hspace{.2in}$\blacktriangleright$] internalist and externalist evaluation
\item[B\hspace{.2in}$\blacktriangleright$] first and second order desires
\item[C\hspace{.2in}$\blacktriangleright$] qualitative and quantitative evaluation
\item[D\hspace{.2in}$\blacktriangleright$] weak and strong evaluation
\end{description}
\end{frame}

\begin{frame}
  \frametitle{Reproaches Against Existentialism}
  \begin{itemize}
  \item quietism (communists) intellectual contemplation leads to just
    another bourgeois worldview
  \item privileging the solitary over solidarity, forgetting the
    ``smile of the infant'' (Catholics)
  \item the seriousness of human affairs and moral responsibility (Christians)
  \end{itemize}
  JPS: They call us gloomy, when they say (dismal proverbs), ``charity
  begins at home''
\end{frame}

\begin{frame}
  \frametitle{Existence Precedes Essence}
  In the 18th century, the idea of the artisan-designer for human
  beings was sidelined, but the quest for a human essence remained.
  The paper knife.
  \begin{itemize}
  \item ``man surges up in the world and defines himself afterwards''
  \item man differs from a scientific object because he is nothing
    until he makes something of himself
  \item in choosing for himself he chooses for all men (Christian
    trade union, monogamy)
  \item man is condemned to be free
  \end{itemize}
\end{frame}

\begin{frame}
  \frametitle{Open Questions}
  \begin{itemize}
  \item How typical is the Free French example for a moral dilemma?
  \item Man fashions both the signs and their interpretation
    $\longrightarrow$ the importance of hermeneutics for ethics
    (Jesuit priest)
  \item Sartre's reliance on Kant (``one ought always to ask oneself
    what would happen if everyone did as one is doing,'' 292) and
    Descartes (``the starting point for truth is one's immediate sense
    of self'', 302)
  \item Contrast human nature with the human condition to understand
    what Sartre means by cowardice. ``Those who hide from total
    freedom, in a guise of solemnity or with deterministic excuses, I
    shall call cowards'' (308). The story of Carlos Flores and the
    subway accident.
  \item How appropriate is it to demand strict authenticity from
    humans? Don't we need an essence for authenticity, for ``being who
    we are''? Michel Foucault's criticism.
  \end{itemize}
\end{frame}

\begin{frame}
  \frametitle{Hermeneutic and Scientific Method}
  \begin{equation}
    \label{eq:paumatae}
    \begin{array}{rcl}
      \mbox{understanding} & \mbox{vs.} & \mbox{explaining} \\
      \mbox{narrative} & \mbox{vs.} & \mbox{model} \\
      \mbox{inter-textuality} & \mbox{vs.} & \mbox{experiment} \\
      \mbox{coherence} & \mbox{vs.} & \mbox{falsifiability} \\
      \mbox{hypostatic} & \mbox{vs.} & \mbox{hypothetical} \\
      \mbox{texts} & \mbox{vs.} & \mbox{nature} \\
      \mbox{integration} & \mbox{vs.} & \mbox{differentiation} \\
      \mbox{dialectic} & \mbox{vs.} & \mbox{monism} \\
    \end{array}\notag
  \end{equation}
\end{frame}

\begin{frame}
  \frametitle{The Hermeneutic Tradition}
\begin{figure}[h]
\includegraphics[scale=.32]{./subob.png}
\end{figure}
\end{frame}

\begin{frame}
  \frametitle{Structuralism}
\begin{figure}[h]
\includegraphics[scale=.3]{./structable.png}
\end{figure}
\end{frame}

\begin{frame}
  \frametitle{Frankfurt's Second-Order Desires}
  \begin{itemize}
  \item first-order and second-order desires
  \item qualitative and quantitative evaluation of desires
  \item weak and strong evaluation (the defining feature for strong
    evaluation is to have a qualitative distinction of the worth of
    the motivations)
  \end{itemize}
Utilitarianism fails on two counts:
\begin{enumerate}
\item weak evaluation is not reducible to calculation
\item moral evaluation is not reducible to weak evaluation
\end{enumerate}
\end{frame}

\begin{frame}
  \frametitle{Criteria for Distinction}
  \begin{description}
  \item[contingency] strong evaluation dilemmas cannot be resolved by appeal to contingencies
  \item[contrast] strong evaluation proceeds on the basis of
    contrasts (courage is meaningless without cowardice and vice
    versa)
  \end{description}
\end{frame}

\begin{frame}
  \frametitle{The Problem with Second-Order Desires}
\begin{itemize}
\item the practical vs the moral approach to raising children
\item consider a law that makes something morally repugnant
  (abortion, drug consumption, corruption, sex with children) more
  legal and less frequent $\longrightarrow$ would you assent to it?
\item are the moral notions at the basis of strong evaluation only
  a front to legitimize moral/social/economic pressure on others?
  (Mandeville)
\item the drug that allows you to eat cake and be healthy as well
  (deflated descriptions vs moral reality)
\item CT resolves this later by appealing to re-evaluation
\end{itemize}
\end{frame}

\begin{frame}
  \frametitle{J.S. Mill's Defence}
  Charles Taylor: ``honour, dignity, integrity are \emph{simply}
  other pleasurable states to which we give \emph{high-sounding}
  names'' (23). Why does reducibility in principle imply undue
  simplicity? Why is the ``simple weigher'' inferior to the
  ``strong evaluator''? Monism can be maintained even in the face
  of certain kinds of emergence.
  \begin{block}{Charles Taylor}
    thus the strong evaluator has articulacy and depth which the
    simple weigher lacks
  \end{block}
Really? Isn't it the monist who makes progress? Perhaps the monist
does not succeed with the reduction, but many productive (rather
than successful) scientific programs originate in a desire to
perform a reduction (alchemy). 
\end{frame}

\begin{frame}
  \frametitle{The Importance of Articulation}
  First-order choices are inarticulable. Second-order choices flow
  from the use of language and articulated coherence. Note the
  following analogy:
  \begin{itemize}
  \item Descartes' \textsc{je pense} solves an epistemological
    problem (skepticism); as a consequence, we perceive ourselves
    primarily and dominantly as thinking beings
  \item Taylor's \textsc{je parle} solves an ethical problem (what
    characterizes moral responsibility); as a consequence, we
    perceive ourselves primarily and dominantly as
    talking/interpreting/story-telling beings
  \end{itemize}
The story of my great-grandparents. 
\end{frame}

\begin{frame}
  \frametitle{Responsibility and Radical Choice}
  The problem with radical choice (Free French, Nepal) is that it
  solves the dilemma by rendering the force of the losing side
  inoperative (this is a problem that Bernard Williams has
  addressed with respect to utilitarianism and Kantian moral
  theory in articles such as ``Consequentialism and Integrity''
  and ``Ethical Consistency'' -- see his concept of ``regret'').

\bigskip

This is the core claim: agents of radical choice are simple
weighers. Taylor wasn't after the utilitarians, he was after the
existentialists, undermining their position by putting them in the
same boat as the utilitarians. The Meursault/Rieux incoherence
problem and what is the object of moral evaluation (in Taylor's
case: the way in which an agent successfully articulates coherence
in their life; in the existentialist's case: the way in which an
agent accepts the human condition).
\end{frame}

\begin{frame}
  \frametitle{Identity}
  \begin{block}{Charles Taylor}
    This is what is impossible in the theory of radical choice.
    The agent of radical choice would at the moment of choice have
    \emph{ex hypothesi} no horizon of evaluation. He would be utterly
    without identity. He would be a kind of extensionless point, a
    pure leap into the void. But such a thing is an impossibility,
    or rather could only be the description of the most terrible
    mental alienation. The subject of radical choice is another
    avatar of that recurrent figure which our civilization aspires
    to realize, the disembodied ego, the subject who can objectify
    all being, including his own, and choose in radical freedom.
    But this promised total self-possession would in fact be the
    most total self-loss. (35)
  \end{block}
\end{frame}

\begin{frame}
  \frametitle{Where Does the Buck Stop?}
Where does the buck stop? Not at radical choice (this would be a
sort of foundationalism), but at articulations (hermeneutic
circle). At the core are not calculations, but interpretations.
There is a hermeneutic circle from self-interpretations that are
constitutive of experience to evaluations of these
self-interpretations. Radical re-evaluation: Quine's web of
beliefs and Neurath's boat.  
\end{frame}

\end{document}
